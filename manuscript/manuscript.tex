\documentclass{article}[12pt]
\usepackage{amsmath, graphicx}
\usepackage[margin=2cm]{geometry}
\begin{document}
\title{On the Simultaneous Use of Fixed Effects on Cases and Time Points}
\date{\today}
\author{Jonathan Kropko\\ University of Virginia \\ jkropko@virginia.edu \and Robert Kubinec \\University of Virginia\\rmk7xy@virginia.edu}
\maketitle
\begin{abstract}
Time-series cross-sectional (TSCS) data contain a sample of cases observed at several repeated time points.  Researchers commonly employ fixed effects (FEs) on the cases to remove cross-sectional unobserved heterogeneity from the model.  Recently, a great deal of applied work also includes FEs on time points with the intention of also accounting for omitted variables in the time dimension.  The properties of the model that includes FEs on both cases and time are not well understood.  We derive the formal two-way FE estimator and show that it does not account for unobserved heterogeneity in the cross-sectional or the time dimension.  We further demonstrate that the two-way FE model is sensitive to whether the panels are balanced while a model that includes FEs only on cases or only on time points is not.  We demonstrate that the choice of model has a profound influence on the findings of an analysis that assesses the relationship between a country's wealth and level of democracy.  We recommend that researchers avoid the two-way FE model, and instead use a model with FEs only on cases for research questions that involve the comparison of time trends, or a model with FEs only on time points for research questions that require the comparison of different cases. 
\end{abstract}
\end{document}